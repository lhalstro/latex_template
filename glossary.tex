
\usepackage[acronym, nogroupskip]{glossaries}  %cite acronyms with \gls{} (or \gls*{} to not link) (automatic), \arclong{} for spelled out, \arcfull{} for spelled out plus abbr, define acronyms with \newacronym{WT}{WT}{Wind Tunnel}
%nogroupskip prevents doubles spacing between letters in acronym glossary for AIAA style

%this sets the link color of all glossary references:
\newcommand\glossarylinkcolor{ultramarineblue}
% \renewcommand*{\glstextformat}[1]{\textcolor{black}{#1}}
% \renewcommand*{\glstextformat}[1]{\textcolor{ultramarineblue}{#1}}
\renewcommand*{\glstextformat}[1]{\textcolor{\glossarylinkcolor}{#1}}



%------------------------------------------------
% NOMENCLATURE GLOSSARY
%-----------------------------------------------

%https://tex.stackexchange.com/questions/386666/how-can-i-get-the-first-use-of-a-glossary-to-show-the-full-description

% \newglossaryentry{alpha}{
%     type=nom,
%     name={a},
%     description={angle of attack},
%     first={angle of attack (a)}
%     }
% \gls{alpha}      --> "angle of attack (a)" then "a"
% \glsdesc{alpha}  --> "angle of attack"
% \Glsdesc{alpha}  --> "Angle of attack"
% \GLSdesc{alpha}  --> "ANGLE OF ATTACK"
% \glsname{alpha}  --> "a"


%make nomenclature-specific glossary
\newglossary{nom}{nom}{nml}{Nomenclature}

%store nomenclature symbol with and without math operators
\glsaddkey
    {eqn}% new key
    {\relax}% default value if "eqn" isn't used in \newglossaryentry
    {\glsentryeqn}% description and eqn (DONT USE THIS)
    {\Glsentryeqn}% analogous to \Glsentrytext
    {\glseqn}% analogous to \glstext
    {\Glseqn}% analogous to \Glstext
    {\GLSeqn}% analogous to \GLStext

%call nomenclature in an equation: `\( \eq{alpha} = 10^\circ \)'
%    (dont color hyperlink because that looks bad)
\newcommand{\eq}[1]{\glseqn*{#1}} %dont hyperlink because we want the equation to be one color
% \newcommand{\eq}[2][]{\glseqn{#2}}

%macro to make new nomenclature entries
\newcommand{\newnom}[3]{
    %args: tag, symbol, description
    %Creates new nomenclature entry
    \newglossaryentry{#1}{
        type=nom,  %add to nomenclature glossary
        name={\(#2\)}, %symbol (with math environment)
        eqn={#2}, %symbol without equation operator, so you can call this in math environment
        description={#3}, %text description
        first={#3 (\(#2\))}, %first usage is "description" (symbol)
        % #1%not sure why the fake first input needs to be here
        }
}

%cite "description (symbol)"
\newcommand{\glsall}[1]{\glsdesc{#1} (\glsname{#1})}
\newcommand{\Glsall}[1]{\Glsdesc{#1} (\glsname{#1})}
\newcommand{\glsallplural}[1]{\glsdescplural{#1} (\glsplural{#1})}



\newglossaryentry{alpha}{
    type=nom,
    name={\(\alpha\)},
    description={angle of attack},
    first={angle of attack (\(\alpha\))},
    eqn={\alpha}
    }

\newnom{t}{t}{time}
\newnom{alphat}{\alpha_T}{total angle of attack}
\newnom{alphadot}{\dot{\alpha}}{rate of angle of attack}
\newnom{beta}{\beta}{side-slip angle}
\newnom{thetas}{\theta_s}{polar angle}
\newnom{phis}{\phi_s}{azimuthal angle}
\newnom{phic}{\phi_c}{aerodynamic clocking angle}
\newnom{phi}{\phi}{roll angle}
\newnom{yaw}{\psi}{yaw angle}
\newnom{theta}{\theta}{pitch/swing angle}
\newnom{thetadot}{\dot{\theta}}{angular velocity}
\newnom{thetadotdot}{\ddot{\theta}}{angular acceleration}
\newnom{theta0}{\theta_{0}}{initial swing angle}
\newnom{thetamax}{\theta_{max}}{maximum swing angle}
\newnom{CA}{C_A}{axial force coefficient}
\newnom{CN}{C_N}{normal force coefficient}
% \newnom{CNm}{C_{N,m}}{missile-frame normal force coefficient}
\newnom{Cd}{C_d}{drag coefficient}
\newnom{Cm}{C_{m}}{pitching moment coefficient}
% \newnom{Cmm}{C_{m,m}}{missile-frame pitching moment coefficient}
\newnom{Mm}{M_{m}}{pitching moment}
\newnom{L/D}{L/D}{lift-to-drag ratio}
\newnom{CD}{C_D}{drag force coefficient}
% \newnom{MYi}{\Sigma M_{Y_i}}{total pendulum moment}
% \newnom{CMYi}{\Sigma C_{M,Y_i}}{total pendulum moment coefficient}
% \newnom{CMZ}{\Sigma C_{M,Z}}{total azimuthal pendulum moment coefficient}
% \newnom{Cdyn}{C_{dyn}}{coefficient increment due to dynamic effects}



% \newnom{Faero}{\vec{F}_{aero}}{net aerodynamic force}
\newnom{FA}{F_A}{axial force}
\newnom{FN}{F_N}{normal force}

\newnom{Re}{Re}{Reynolds number}
\newnom{M}{M}{Mach number}
\newnom{St}{St}{Strouhal number}
\newnom{P}{P}{pressure}
\newnom{Cp}{C_P}{pressure coefficient}
\newnom{Cp0}{C_{P_0}}{stagnation pressure coefficient}
\newnom{q}{\overline{q}}{dynamic pressure}
\newnom{Cv}{C_v}{thermal energy coefficient}
\newnom{rhoinf}{\rho_\infty}{freestream density}
\newnom{Vinf}{V_\infty}{freestream velocity}
\newnom{Veff}{V_{eff}}{effective velocity}
\newnom{Vn}{V_{N}}{tangential swing velocity}
\newnom{alphaeff}{\alpha_{eff}}{effective angle of attack}
\newnom{rho}{\rho}{density}
\newnom{V}{\vec{V}}{velocity}
\newnom{Vmag}{V_{mag}}{total velocity magnitude}
\newnom{g}{\vec{g}}{gravitational constant}
\newnom{m}{m}{mass}
\newnom{I}{I}{moment of inertia}
\newnom{CT}{C_T}{thrust coefficient}

\newnom{CO2}{\text{CO}_2}{carbon dioxide}

\newnom{mu}{\mu}{dynamic viscosity}
\newnom{uein}{\vec{u}}{Einstein notation velocity}
\newnom{xein}{\vec{x}}{Einstein notation dimension}
\newnom{grad}{\vec{\nabla}}{gradient function}

% \newnom{}{d}{diameter}
% \newnom{}{h}{Height}
\newnom{vec}{\vec{(\,)}}{vector quantity}
\newnom{mean}{\overline{(\,)}}{mean quantity (time-averaged)}
\newnom{fs}{(\,)_{\infty}}{freestream quantity}
\newnom{pert}{(\,)'}{perturbation quantity}
\newnom{perD}{(\,)_D}{diameter-based reference length}
\newnom{eini}{(\,)_i}{first Einstein notation index }
\newnom{einj}{(\,)_j}{second Einstein notation index }
% \newnom{}{()_c}{Chord-based reference length}

\newnom{chuteframe}{|_{_{P}}}{parachute reference frame}


\newnom{dswall}{\Delta s_w}{wall spacing}
\newnom{muwall}{\mu_w}{wall viscosity}
\newnom{rhowall}{\rho_w}{wall density}
\newnom{utau}{u_\tau}{friction velocity}
\newnom{tauwall}{\tau_w}{wall shear stress}




%theta_s
%CA
%CN, CNm, Cm,
%2D, 3D


%from turb lit study:
% $M$   & Mach number, N.D. \\
% $Re$   & Reynolds number, N.D. \\
% $\alpha$ & Angle of attack, deg\\
% $\rho$ & Density, $kg/m^3$\\
% $P$   & Pressure, $Pa$ \\
% $\mu$   & Dynamic viscosity, $Pa \cdot s$ \\
% $t$   & Time, $s$ \\
% $V$   & Velocity, $m/s$ \\
% $u$   & Einstein notation velocity, $m/s$ \\
% $x$   & Einstein notation dimension, $m$ \\
% $\nabla$   & Gradient function \\
% $C_d$   & 2-Dimensional drag coefficient, N.D. \\
% $d$   & Diameter, $m$ \\
% $h$   & Height, $m$ \\
% \multicolumn{2}{@{}l}{Subscripts}\\
% $()_{\infty}$ & Freestream quantity\\
% $\vec{()}$ & Vector quantity\\
% $\overline{()}$ & Mean quantity (time-averaged)\\
% $()'$      & Perturbation quantity\\
% $()_D$     & Diameter-based reference length\\
% $()_c$     & Chord-based reference length\\
% $()_i$     & Einstein notation index \\
% $()_j$     & Einstein notation index \\







%------------------------------------------------
% ACRONYM GLOSSARY
%-----------------------------------------------


% \setabbreviationstyle[acronym]{long-short}

% \setacronymstyle{long-short-desc} %use this if you want to include descriptions as well. descriptions in glossary with `altlist'

\newcommand{\newacr}[2]{\newacronym{#1}{#1}{#2}}%save time with duplicate entry
\newacr{CFD}{Computational Fluid Dynamics}
\newacr{NASA}{National Aeronautics and Space Administration}
\newacr{AETC}{Aerosciences Evaluation and Test Capabilities}
\newacr{STMD}{Space Technology Mission Directorate}
\newacr{GCDP}{Game Changing Development Program}
\newacr{DSS}{Descent System Study}
\newacr{GMP}{Geometry Manipulation Protocol}
\newacr{XML}{Extensible Markup Language}
\newacr{DCF}{Domain Connectivity Function}
\newacr{CGT}{Chimera Grid Tools}
\newacr{MPCV}{Multi-Purpose Crew Vehicle}
\newacr{HLLC}{Harten-Lax-van Leer-Contact}
\newacr{HLLE++}{Harten-Lax-van Leer-Einfeldt}
\newacr{SSOR}{Symmetric Successive Over-Relaxation}
\newacr{RANS}{Reynolds-averaged Navier-Stokes}
\newacr{URANS}{Unsteady Reynolds-averaged Navier-Stokes}
\newacr{SST}{Shear Stress Transport}
\newacr{CC}{compressibility correction}
\newacr{RC}{rotation and curvature}
\newacr{QCR}{Quadratic Constitutive Relation}
\newacr{SA}{Spalart-Allmaras}
\newacr{DES}{Detached Eddy Simulation}
\newacr{LES}{Large Eddy Simulation}
\newacr{TKE}{turbulence kinetic energy}
\newacr{DNS}{Direct Numerical Simulation}
\newacr{SGS}{Subgrid-Scale model}
\newacr{FFT}{fast Fourier transform}
\newacr{PSD}{Power Spectral Density}
\newacr{ODE}{Ordinary Differential Equation}
\newacr{WTT}{wind tunnel test}
\newacr{NFAC}{National Full-scale Aerodynamics Complex}
\newacr{LUPWT}{Langley Unitary Plan Wind Tunnel}
\newacr{AMR}{Adaptive Mesh Refinement}
\newacr{EDL}{Entry, Descent, and Landing}
\newacr{FSI}{Fluid-structure Interaction}
\newacr{IBM}{Immersed Boundary Method}
\newacr{IMU}{Inertial Measurement Unit}
\newacr{FBC}{Forward Bay Cover}
\newacr{CG}{Center of Gravity}
\newacr{COM}{Center of Mass}
\newacr{CPU}{central processing unit}
\newacr{HRVIP}{Human/Vehicle/Robotic Integration and Performance}
\newacr{CPAS}{Capsule Parachute Assembly System}
\newacr{CALA}{Canopy Loads Analysis}
\newacr{PTV}{Parachute Test Vehicle}
\newacr{DGB}{Disk-Gap-Band}
\newacr{EDU}{Engineering Design Unit}
\newacr{FBCP}{Forward Bay Cover Parachute}
\newacr{SRP}{supersonic retropropulsion}
\newacr{HIAD}{Hypersonic Inflatable Aerodynamic Decelerator}

\newacr{BC}{boundary condition}



%SSOR, HLLC, GUI, DoF



\newcommand\nfac{\gls{NFAC} \(80^\prime\textrm{x}120^\prime\)~} % "NFAC 80'x120'"





%I dont want a glossary for AIAA, just acronym tracking
    %but I want glossary in thesis, so use these macros to keep track
% %standard functionality:
% \newcommand\myglsstr{\gls*}
% \newcommand\mygls{\gls}
% \newcommand\myacrfull{\acrfull}
% \newcommand\myacrlong{\acrlong}
% no linking:
\newcommand\myglsstr{\gls*}
\newcommand\mygls{\gls}
\newcommand\myacrfull{\acrfull}
\newcommand\myacrlong{\acrlong}







%AIAA glossary table formatting (setting to specify with  \printglossary)
\usepackage{longtable,tabularx}
\setlength\LTleft{0pt}
\newglossarystyle{aiaa}{%
\setglossarystyle{long}% base this style on the long style
\renewenvironment{theglossary}{%
    \begin{longtable*}{@{}l @{\quad=\quad} l@{}}}%
{\end{longtable*}}%
}

% \setglossarystyle{mylong} %set all glossaries to this style (or do individually in printglossaries)



%---------------------------------------------
% MAKE THE GLOSSARIES WORK:

%If glossary doesn't appear, this command isn't running. To fix, execute `makeglossaries docname' in the main directory (where your latex document is `docname.tex')
\makeglossaries

% %spell out acronyms at the beginning of every chapter (makes sense for a long document like a thesis)
% \preto\chapter{\glsresetall}
