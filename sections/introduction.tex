\section{Introduction}\label{sec:intro}


\lettrine{E}{ach} sentence of the document should be on one line to improve version control.
You can use acronyms from the glossary like \gls{NASA}.
They will be spelled out in their first use, then abbreviated there after like this: \gls{NASA}.
By default, it will re-expand acronyms in the main body even if they were used in the abstract (You can also add behavior to redefine acronyms at each section).
You can spell out an acronym at any time like this: \acrfull{NASA}.

Basic citations look like this~\cite{halstrom2021aerodynamic}.
Multi-citations look like this~\cite{halstrom2022overflow,halstrom2021investigation} or this~\cite{halstrom2022overflow,halstrom2021investigation,halstrom2021aerodynamic}.
Also do, citeall, citeauthor, etc.



Manual process to compile:
\begin{itemize}
  \item \textit{pdflatex main.tex (tx main.tex)}
  \item \textit{bibtex main.aux}
  \item \textit{makeglossaries main}
  \item \textit{tx main.tex}
\end{itemize}


\noindent List of examples to add:
\begin{itemize}
  \item Examples of how to use each kind of citation macro (see google doc notes) %\cite: [num], \citet: Author [num], \citeauthor, \citeyear, \citeall (defined below)
  \item also show all macros for glossary entries
  \item explain how to change color of glossary lines (in the top of glossary.tex)
  \item Examples of how to format numbers in text
  \item table of font sizes
  \item how to convert to word (open compiled pdf with word)
  \item examples of subfigures
\end{itemize}

We can also reference figures, like the comparison of two vehicles (a low-\glsname{L/D} \gls{HIAD}~\cite{halstrom2022overflow} and mid-\glsname{L/D} CobraMRV vehicle~\cite{halstrom2022overflow,halstrom2021investigation,halstrom2021aerodynamic}), which can be found in Figure~\ref{fig:vehicleconcepts}.
An in-line math example: \(M \sim 2\).

\begin{figure}[htb] %choose best location in document
% \begin{figure}[H] %force figure to this exact location
\begin{center}
    \newcommand\picW{0.49\textwidth}
    \newcommand\picH{0.75\textwidth}
    \begin{subfigure}{\picW}
      \centering
        %[trim=left bottom right top, clip]
        \includegraphics[trim=0 0 0 0, clip, height=\picH]{dummy}
        \caption{ Low-L/D HIAD }
        \label{fig:hiadconcept}
    \end{subfigure}
    % \hspace*{\fill} %separation between subfigures
    \begin{subfigure}{\picW}
      \centering
      \includegraphics[trim=0 0 0 0, clip, height=\picH]{Images/dummy.png}
        \caption{ Mid-L/D CobraMRV }
        \label{fig:cobraconcept}
    \end{subfigure}
    \caption{ Comparison of different vehicles (shows how to force images to a given height)~\cite{halstrom2021aerodynamic} }
    \label{fig:vehicleconcepts}
\end{center}
\end{figure}

You can also reference tables, like Table~\ref{tab:tableexample}.

% \begin{table}[htb] %choose best location in document
\begin{table}[H] %force figure to this exact location
\begin{center}
\caption{Example table of test conditions}
\label{tab:tableexample}
\begin{tabular}{r | l}
Geometry      & \(1A, 1B, 1C, 1D, \boxed{1E, 1F}, 2A\) \\
\((M,Re/ft)\) & \((1.000,1e1), (2.000,1e2), (3.000,1.5e2)\) \\
\(\alpha\)    & \(\degr{0}, \degr{1}, \degr{2}\) \\
\(C_T\)       & \(0.0, 0.1, 0.9, 1.0\) \\
\end{tabular}
\end{center}
\end{table}




Here is the last paragraph.
You can reference other parts of the paper, such as Section~\ref{sec:conclusion}, which is the conclusion.




TESTING: Reference with https link in doi field~\cite{ref1}.
Reference with just doi in doi field~\cite{ref2}.
